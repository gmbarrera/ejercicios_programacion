\chapter{Instrucciones condicionales}

\section{Condicional Simple}

 \begin{enumerate}[resume]
  
	\item Leer desde teclado 2 números y mostrar el mayor de ellos.
  
	\item Que pida un número del 1 al 7 y diga el día de la semana correspondiente
  
	\item Que pida un número del 1 al 12 y diga el nombre del mes correspondiente
   
	\item Que pida una letra y detecte si es una vocal.
    
    \item Simular la tirada de un dado e indicar si es par o impar.
    
	\item (15’) Ingresar dos números y multiplicarlos. Mostrar una leyenda según el producto sea negativo, positivo o cero.
    
   \item Leer 4 números y mostrar el mayor.
   
   \item Leer 3 números y mostrarlos ordenados.
     
   \item Leer 2 números y si es posible dividir el primero por el segundo.
    
   \item Leer 4 números, calcular el promedio y determinar cuántos números son mayores al promedio.
    
   \item Leer un entero A y dos reales B y C. Si A es mayor al promedio entre B y C, mostrar A*B caso contrario mostrar A*C
    
   \item Indicar el valor final de la compra, si solo se puede comprar un solo tipo de producto y si la cantidad comprada supera los \$100, se le debe realizar un descuento del 10\%  sobre el total de la compra.    
    
   \item Generar 1 número aleatorio entre 1 y 5, almacenarlo en la variable “num”. Luego
generar tantos números aleatorios como contenga “num” y calcular el promedio.

   \item Leer desde teclado un número entero, si dicho número es menor a 100 generar un
 número aleatorio real entre 10 y 50, y mostrarlo. En cambio si no es menor a
 100 generar un número aleatorio entero entre 1 y 20.

   \item Un obrero necesita calcular su salario semanal, el cual se obtiene de la
   siguiente manera (leer la cantidad de horas trabajadas): Si trabaja 40 horas
   o menos se le paga \$16 por hora Si trabaja más de 40 horas se le paga \$16
   por cada una de las primeras 40 horas y \$20 por cada hora extra.

  \item Pedir una nota entre 0 y 10, y mostrar en forma de palabras la
    calificación obtenida:
    \begin{itemize}
      \item de 0 a 4: insuficiente
      \item 5: suficiente
      \item 6: bien
      \item 7 y 8: notable
      \item 9: sobresaliente
      \item 10: excelente
    \end{itemize}

  \item Pedir el día, mes y año e indicar si es una fecha valida (Leerlos como
    números)
    \begin{enumerate}
      \item Suponiendo que todos los meses tienen 30 días
      \item Tomando las longitudes de meses reales\\
          31 días: 1 Enero- 3 Marzo- 5 Mayo- 7 Julio- 8 Agosto-10 Octubre-12 Diciembre\\
          30 días: 4 Abril– 6 Junio– 9 Septiembre– 11 Noviembre\\
          28 días: 2 Febrero, año no bisiesto\\
          29 días: 2 Febrero, año bisiesto
      \end{enumerate}

  \item Pedir un numero de 5 cifras (validarlo), determinar si es capicúa
    (TIP: operador %)

  \item Pedir 4 números, calcular y mostrar la diferencia entre el mayor y el
    menor.

  \item Pedir la fecha de nacimiento (día, mes y año) y determinar el signo del
    zodíaco.
    \begin{enumerate}
      \item Capricornio: enero 21 / febrero 16
      \item Acuario: febrero 17 / marzo 12
      \item Piscis: marzo 13 / abril 18
      \item Aries: abril 19 / mayo 14
      \item Tauro: mayo 15 / junio 21
      \item Géminis: junio 22 / julio 20
      \item Cáncer: julio 21 / agosto 10
      \item Leo: agosto 11 / septiembre 16
      \item Virgo: septiembre 17 / octubre 31
      \item Libra: noviembre 1 / noviembre 23
      \item Escorpio: noviembre 24 / noviembre 29
      \item Ofiuco: noviembre 30 / diciembre 18
      \item Sagitario: diciembre 19 / enero 20
    \end{enumerate}

  \item Un postulante a un empleo, realiza un test de capacitación, se obtuvo
    la siguiente información: cantidad total de preguntas que se le realizaron
    y la cantidad de preguntas que contestó correctamente. Se pide confeccionar
    un programa que ingrese los dos datos por teclado e informe el nivel del
    mismo según el porcentaje de respuestas correctas que ha obtenido, y
    sabiendo que:
      \begin{itemize}
        \item Nivel máximo:              Porcentaje$>$=90\%.
        \item Nivel medio:               Porcentaje$>$=75\% y $<$ 90\%.
        \item Nivel regular:             Porcentaje$>$=50\% y  $<$ 75\%.
        \item Fuera de nivel:            Porcentaje$<$50\%.
      \end{itemize}

  \item Construir un programa que calcule el índice de masa
    corporal de una persona e indique el estado
    en el que se encuentra esa persona en función del valor de IMC:
    \[IMC = peso [kg] / altura^2 [m] \] 
    \begin{table}[H]
     \begin{center}
      \begin{tabular}{l|l}
        Valor de IMC &    Diagnóstico\\
        \toprule
        $<$ 16         &      Criterio de ingreso en hospital\\
        de 16 a 17   &   infrapeso\\
        de 17 a 18   &   bajo peso\\
        de 18 a 25   & peso normal (saludable)\\
        de 25 a 30   &   sobrepeso (obesidad de grado I)\\
        de 30 a 35   &   sobrepeso crónico (obesidad de grado II)\\
        de 35 a 40   &   obesidad premórbida (obesidad de grado III)\\
        $>$40          &       obesidad mórbida (obesidad de grado IV)\\
      \end{tabular}
     \end{center}
    \end{table}       

  \item Construir un programa que calcule y muestre por pantalla
    las raíces de la ecuación de segundo grado de coeficientes reales (\ref{cuadratica}). El
    programa debe diferenciar los diferentes casos que puedan surgir: la
    existencia de dos raíces reales distintas, de dos raíces reales iguales y
    de dos raíces complejas.  
    \begin{equation}
    \label{cuadratica}
    x =\frac{ -b \pm \sqrt{b^2-4ac}}{2a}
    \end{equation}
    
    Nota: se recomienda el empleo de sentencias if–else anidadas.
     
  \item Hacer un algoritmo que tome el peso en libras de una
    cantidad de ropa a lavar en una lavadora y nos devuelva el nivel
    dependiendo del peso; además nos informe la cantidad de litros de agua que
    necesitamos. Se sabe que con más de 30 libras la lavadora no funcionara ya
    que es demasiado peso. Si la ropa pesa 22 ó más libras, el nivel será de
    máximo; si pesa 15 o más nivel será de alto; si pesa 8 ó más será un nivel
    medio o de lo contrario el nivel será mínimo. La cantidad de litros de agua
    necesaria es el 25\% del peso de la ropa.
     
  \item Martha va a realizar su fiesta de quince años. Por lo cual
    ha invitado a una gran cantidad de personas. Pero también ha decidido
    algunas reglas: Que todas las personas con edades mayores a los quince
    años, sólo pueden entrar si traen regalos; que jóvenes con los quince años
    cumplidos entra totalmente gratis pero los de menos de quince años no
    pueden entrar a la fiesta. Hacer un algoritmo donde se tome la edad de una
    persona y que requisito de los anteriores le toca cumplir si quiere entrar.
     
  \item El ejército nacional colombiano ha decidido hacer una
    jornada de ventas de libretas militares para muchos hombres que no han
    definido su situación militar u otros que no son aptos para prestar el
    servicio. Además de la edad de joven, se tendrá en cuenta el nivel del
    sisben de la persona. Para todos los hombres mayores de 18 años la libreta
    tendrá un costo de \$350000 pero para aquellos que tengan nivel 1, se les
    hará un descuento del 40\%; para los de nivel 2, el descuento será del 30\%;
    para nivel 3 del 15\%; y para los demás estratos o niveles no habrá
    descuento. Para los jóvenes con los 18 años la libreta tiene un costo de
    \$200000 y los jóvenes con nivel del sisben 1, tendrán un descuento del 60\%;
    para los de nivel 2, descuento del 40\%, para los del 3, un descuento del
    20\% y para los demás estratos no habrá descuento. Hacer un algoritmo que
    tome la edad y el nivel del sisben de un hombre y nos muestre descuento que
    le hacen y su valor final a pagar.
     
  \item Se cuenta con los votos obtenidos por Juan, Pedro y María
    en una elección “mejor compañero del aula”. Para ganar la elección se debe
    obtener como mínimo el  50\% del total de votos más 1.
    En caso que no haya un ganador, se repite la elección en una  segunda
    vuelta, yendo a ésta los dos candidatos que obtengan la más alta votación.
    Se anula la  elección en caso de producirse un empate doble por el segundo
    lugar o un empate triple.
    Diseñe  un algoritmo que determine el resultado de la elección.
     
  \item Realizar un juego  que muestre en pantalla una lista de
    personajes famosos:
    Olmedo, Carrio, Susana, Maradona, Menem
    El usuario debe seleccionar uno de ellos y luego contestar las siguientes
    preguntas, en este orden
    -          Está vivo?
    -          Es mujer?
    -          Es político?
    La computadora debe indicarle de acuerdo a las respuestas, cual es el
    personaje elegido por el usuario
 
  \item Que muestre un menú donde las opciones sean “Equilátero”,
    “Isósceles” y “Escaleno”, pida una opción y calcule el perímetro del
    triángulo seleccionado
     
  \item Que pase de Kg a otra unidad de medida de masa, mostrar en
    pantalla un menú con las opciones posibles (Gramos, miligramos, decigramos,
    centigramos)
     
  \item Una ciudad se divide en tres zonas: 1, 2 y 3. En la zona
    1, el metro cuadrado se cotiza a \$1000. En la zona 2, se cotiza a \$900, y
    en la zona 3 a \$700. Una vez calculada la cotización, si esta supera el
    monto de \$80000, se le aplica un 2\% de impuesto para obtener el precio
    final del inmueble. Obtener el precio final de un inmueble si se ingresan
    la zona y la superficie del mismo.
     
  \item Dada el peso, la altura y el sexo, de unos estudiantes.
    Determinar la cantidad de vitaminas que deben consumir estos estudiantes,
    en base al siguiente criterio:
     
    "Si son varones, y su estatura es mayor a 1.60, y su peso es mayor o igual
    a 150 lb, su dosis, serán: 20\% de la estatura y 80\% de su peso. De lo
    contrario, la dosis será la siguiente: 30\% de la estatura y 70\% de su peso.
    Si son mujeres, y  su estatura es mayor de a 1.50 m y su peso es mayor o
    igual a 130 lb, su dosis será: 25\% de la estatura y 75\% de su peso. De lo
    contrario, la dosis será: 35\% de la estatura y 65\% de su peso. La dosis
    debe ser expresada en gramos."
     






\end{enumerate}



\section{Condicional M\'ultiple}

\begin{enumerate}[resume]

	\item Se lee un número entre 1 y 5, luego se ingresa el idioma deseado (1-Español, 2-Ingles). Mostrar cómo se escribe el número ingresado en el idioma seleccionado.

    \item Que pase de Kg a otra unidad de medida de masa, mostrar en pantalla un menú con las opciones posibles (Gramos, miligramos, decigramos, centigramos)

	\item El costo de las llamadas internacionales depende de la zona geográfica del país destino y de la cantidad de minutos hablados. En la siguiente tabla se representa el costo del  minuto por zona, cada zona tiene un número clave conocido por el operador y si el costo supera los 15\$, se le realiza un 15\% de descuento ,solo se ingresa la clave y los minutos El programa debe indicar la zona y el total a pagar

    \begin{table}[H]
     \begin{center}
      \begin{tabular}{l|l|l}
		Clave &           Zona & Precio \\
		\toprule
		12 & América del Norte &  0.68 \\
		15 & América Central   &  0.55 \\
		18 & América del Sur   &  0.62 \\
		19 & Europa            &  0.85 \\
		23 & Asia              &  0.89 \\
		25 & África            &  0.82 \\     
      \end{tabular}
     \end{center}
    \end{table}

	\item Calcular el sueldo de un operario teniendo en cuenta la cantidad de horas extras y la cantidad de horas, de acuerdo a los valores indicados:
    \begin{table}[H]
     \begin{center}
      \begin{tabular}{c|r|r}	 
	 
		CATEGORÍA       &        PRECIO HORA      &         PRECIO  HORA EXTRA \\
		\toprule
		1 & 14\$ & 20.50\$ \\
		2 & 17\$ & 24\$ \\
		3 & 21\$ & 34\$ \\
	 \end{tabular}
   \end{center}
  \end{table}

	Cada trabajador puede tener como máximo 30 horas extras, si tiene más se le paga un 7\% menos del valor indicado en la tabla a cada categoría
	
	\item Se pide determinar si los alumnos pueden acceder a la beca o no de acuerdo a su promedio:
	 \begin{table}[H]
	      \begin{center}
	       \begin{tabular}{l|c|c}
	              Carrera                   &               cuatrimestre                  &      promedio \\
	         \toprule
	1- Economía o 4- contabilidad & $\geq$ 6 &  $>$9.0 \\
	2- Informática o 6- sistemas  & $>$ 6 &  $>$9.2 \\
	3- Agronomía o 5- química     & $>$ 5 &  $>$8.8 \\

		\end{tabular}
	   \end{center}
	  \end{table}
	  
	  \item Una empresa utiliza el siguiente tipo de comercialización para sus productos. Se tienen 2 productos A y B, el producto A tiene tres tipos de fragancias diferentes (primavera, marino y otoño) y el producto B tiene tres tipos de presentaciones diferentes (pequeño, mediano y grande)
	   
	  El costo de cada uno es:
	   \begin{table}[H]
	  	\begin{center}
	  	 \begin{tabular}{l|l}
			PRODUCTO    A  & PRODUCTO  B \\	
            \toprule
	           Otoño  1.50\$         &     grande 2.05\$ \\
	           Primavera 1.55\$      &     mediano  1.60\$ \\
	           Marino 1.60\$         &     pequeño  1.10\$ \\
	   	 \end{tabular}
	   	\end{center}
	   \end{table}
		   	 
	  Indicar el valor final de la compra, si solo se puede comprar un solo tipo de producto y si la cantidad comprada supera los \$100, se le debe realizar un descuento del 10\%  sobre el total de la compra.
	   
	   
	\item Simular la tirada de un dado e indicar si es par o impar, usando Switch
	   
	\item Que pase de Kg a otra unidad de medida de masa, mostrar en pantalla un menú con las opciones posibles (Gramos, miligramos, decigramos, centigramos)
	   
	\item Pedir una nota entre 0 y 10, y mostrar en forma de palabras:
		\begin{itemize}
	       \item de 0 a 4: insuficiente
	  	   \item 5: suficiente
	  	   \item 6: bien
	       \item 7 y 8: notable
	       \item 9: sobresaliente
	       \item 10: excelente
	   \end{itemize}
	   
	\item Escribir y ejecutar un programa que simule un calculador simple, lee dos enteros y un carácter. Si el carácter es “+”, se imprime suma; si es un “-“, se Imprime diferencia; si es “*” se imprime producto, si es “/” se imprime cociente; y si es un “\%” se imprime el resto.
	  
	  
	  
	  
	  
	  
	  
	  
\end{enumerate}
