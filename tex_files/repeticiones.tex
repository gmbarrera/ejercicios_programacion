 \chapter{Instrucciones de repetición}


 \section{Instrucción: while}

 \begin{enumerate}[resume]

   \item Ingresar números hasta que el último sea cero. Calcular la cantidad de positivos.

   \item Leer número enteros mientras sean mayores a cero, calcular si es posible el promedio de los números ingresado.
 
   \item Leer números mientras que sean distintos de cero, de los ingresados determinar el mayor y menor número ingresado.

   \item Una empresa desea conocer el total de sueldos de sus empleados. Si el último sueldo viene cargado con 0, mostrar dicho total.
   
   \item Leer números reales mientras el que se ingresa sea distinto al anterior. Determinar y mostrar el mayor y menor número ingresado.

   \item  Leer un número. Leer otros números mientras sean distintos del primero ingresado, con esos números:
   \begin{itemize}
      \item Determinar y mostrar cuantos son mayores al primer número.
	  \item El promedio de los números que sean múltiplos del primero.
   \end{itemize}	
	
	\item En una elección se cargan documento hasta que uno sea menor o igual a cero. A cada documento se asocian los votos que esa persona puede aportar.
	\begin{enumerate}
		\item ¿Cuál es el promedio de votos por persona?
		\item ¿Qué persona tiene la mayor cantidad de votos?	
	\end{enumerate}
		
	\item Se leen de a 3 valores que representan datos estadísticos, el ingreso de datos finaliza cuando los 2 primeros números de la  terna son ambos 0. Al finalizar el sistema debe determinar y mostrar el mayor de los promedios de las ternas ingresadas. Y la menor suma de valores de ternas.    
	
	\item Leer números mientras sean distintos de cero, contar cuantos pares e impares se ingresaron.
 
    \item Leer un número X, luego leer números mientras estos sean distintos de X. Con esos números sumar la diferencia entre X y cada número ingresado.
 
	\item Leer pares de números (de a 2) mientras la diferencia entre ellos sea mayor a cero. Determinar la mayor diferencia que se registró.
  
    \item Un transportista desea saber cuándo consume en promedio en sus viajes por el país. Además necesita conocer cuál de sus  viajes fue el más largo y el promedio de consumo de combustible. Para ello carga al sistema los siguientes datos: km recorridos y  combustible (por cada viaje). La carga finaliza cuando los km es cero.
  	
	\item Leer números mientras sean distintos de 5, con esos números hacer lo siguiente:
	\begin{enumerate}
  		\item Contar los mayores a 5
		\item Contar cuantos son pares
		\item Sumar aquellos que se ingresaron en orden (posición) par
	\end{enumerate}
 
   Ejemplo Ingreso: \\
                  9       2       3       -1      7       0       -3      10     5 \\
         Salida: \\
                   Se contaron 3 mayores a 5. \\
                   Se contaron 3 pares. \\
                   La suma da 11 \\

 
   \item Una empresa necesita un sistema para poder controlar la facturación que realizan los vendedores. 
		Ingresan los códigos de vendedores (Mayores a cero), cuantas facturas realizaron y los importes de dichas facturas. 
		La empresa necesita saber: 
		\begin{enumerate}
			\item cuál vendedor hizo más ventas; 
			\item cuál fue la factura más alta y 
			\item el promedio general de ventas.
		\end{enumerate}

    \item Se toman muestran del comportamiento de diversos gases en un ambiente controlado. Cada muestra está compuesta de los siguientes datos:
	    \begin{itemize}
            \item Código de la muestra (Mayor a cero)
            \item Densidad (entre 0 y 1)
            \item Peso molecular (mayor a cero)
            \item Tipo de gas (1, 2, o 3)
        \end{itemize}
        Según el tipo de gas se calcula su velocidad de propagación:
            
            $VP1 = densidad *  peso^2 $ 
            
            $VP2 =  peso^4 / densidad $

            $VP3 = densidad^3 / peso^2 $
 
            La carga de datos finaliza cuando el código es cero.
            
            Se pide:
            \begin{enumerate}
            	\item La mayor velocidad registrada.
            	\item El promedio del peso.
			\end{enumerate}


\end{enumerate}





\section{Instrucción: for}

 \begin{enumerate}[resume]

   \item Leer 10 Números y mostrar la suma de ellos
   
   \item Leer 10 Números y mostrar el mayor de ellos
   
   \item Escribir un programa que sume los números comprendidos entre 1 y un valor que se introduce por teclado
   
   \item Dados n números enteros, calcular el menor de ellos.
   
   \item Generar al azar 20 números en el intervalo (0-100), imprimir la cantidad de ceros que salieron
   
   \item Se ingresa la altura (en metros) de \textbf{N} participantes de un grupo de voley. Mostrar la altura máxima
   
   \item Mostrar los múltiplos de 3 entre 3 y N, siendo N ingresado por teclado.
   
   \item Dados \textbf{n} números enteros calcular el rango. Rango: es la diferencia entre el número mayor y el menor.
   
   \item Generar 1 número aleatorio entre 1 y 5, almacenarlo en la variable
    “num”. Luego generar tantos números aleatorios como contenga “num” y
    calcular el promedio de los números ingresados.

   \item Una empresa debe realizar el inventario, para ello recibe una lista con 100 facturas a contabilizar. De cada renglón de la lista se informan el código de artículo, la cantidad y el costo. La empresa debe informar el valor total del inventario y el artículo con costo más caro.

   \item Se leen 25 números entre 0 y 1 que representan la calidad del suelo, cada valor se asocia a un código estudio (mayor a cero)
   \begin{enumerate}   
         \item ¿Cuál estudio tiene la mayor calidad de suelo?
         \item De los mayores a 0.5, ¿cuál es su promedio?
   \end{enumerate}
   
   \item Generar 100 números al azar y verificar la cantidad que salieron entre 0 y 25, la cantidad entre 25 y 50, la cantidad entre 50 y 75 y la cantidad entre 75 y 100.
       
	\item Ingresar 100 número enteros. Calcular y mostrar:
	\begin{enumerate}
	\item El máximo de los números que sean múltiplos del primer número ingresado.
	\item La suma de los números pares.
	\item El promedio de los números ingresados.
	\end{enumerate}
   
   \item Escribir un programa que calcule el valor del término de la sucesión de Fibonacci correspondiente a un número entero determinado.
   \begin{itemize}
      \item a(1) = 1
      \item a(2) = 1
      \item a(3) = 2
      \item a(4) = 3
      \item a(5) = 5
      \item a(6) = 8
      \item a(7) = 13
      \item ...
      \item a(n) = a(n - 1) + a(n - 2)
   \end{itemize}

   \item Dado el ingreso mensual de \textbf{n} personas, se pide obtener:
    \begin{enumerate}
       \item la cantidad de desocupados, la cantidad que cobran hasta 500\$, la cantidad que cobran entre 500 y 1000, la cantidad entre 1000 y 2000, y los que superan los 2000
	   \item el sueldo máximo y el nombre de la persona de sueldo máximo
       \item total de sueldos pagados
   \end{enumerate}
  
   
   
   
\end{enumerate}

