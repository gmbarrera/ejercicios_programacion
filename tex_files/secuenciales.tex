\chapter{Instrucciones Secuenciales}

\begin{enumerate}[resume]

	\item Calcular el sueldo de una persona, conociendo la cantidad de horas que trabaja en el mes y el valor de la hora. (TE: 10')

	\item Una persona compra una heladera de pesos X y por pagar en efectivo le hacen el 10\% de descuento ¿Cuánto abona? (TE: 12')


  \item Crear una variable asignarle un valor a "gusto"  y hacer lo siguiente:
    \begin{enumerate}
      \item Mostrar el doble de ese n\'umero
      \item Almacenar en otra variable el triple más 2 del valor.
      \item Mostrar el resto de la división entera entro el primer número y 5.
      \item En otra variable almacenar la diferencia entre los contenidos de las variables anteriores.
    \end{enumerate}

  \item Crear las variables reales alto, ancho y profundidad. A cada variable asignarle un valor real. Calcular y mostrar:
    \begin{enumerate}
      \item La superficie que se representa. (alto x ancho)
      \item El volumen suponiendo un prisma. (alto x ancho x prof)
      \item El promedio de los tres valores. (al + an + pro) / 3
    \end{enumerate}

  \item Crear 4 variables reales y 2 enteras.
    \begin{enumerate}
      \item Con ellas calcular el promedio de los reales.
      \item Multiplicar la suma de los reales con el primer entero.
      \item Dividir el promedio de los reales por la suma de los enteros.
    \end{enumerate}

  \item Declarar 2 variables con valores enteras, mostrarlas, intercambiar sus valores
y volver a mostrar. Generar 2 algoritmos distintos para conseguirlo.

Ejemplo:         x = 5    y = 8

Luego de intercambiar

x = 8    y = 5
                         
                         



\end{enumerate}
