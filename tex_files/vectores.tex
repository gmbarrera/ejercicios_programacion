\chapter{Vectores}

 \begin{enumerate}[resume]

\item Leer 50 numeros, una vez finalizada la carga de los 50 números, mostrar todos los numeros. (no usar 50 variables)


\item Leer 10 numeros y guardarlos en un vector, luego:
	\begin{enumerate}
	\item Calcular la suma de los valores
	\item buscar el mayor
	\item copiar el vector a otro.
	\end{enumerate}

\item Cargar un vector con los nombres de 10 personas. Luego preguntarle el nombre a alguien y determinar si está en el vector.

\item Leer un vector con 50 numeros. Luego, separar los valores en 2 vectores, uno con los pares y otro con los impares.

\item Leer un vector con 20 numeros:
	\begin{enumerate}
	\item Invertirlo en otro vector.
	\item Invertirlo en si mismo.
	\end{enumerate}

\item  Un sistema controla el peso que tiene un puente, se sabe que a lo sumo caben 10 vehículos en el puente a la vez, a partir de esa cantidad cuando va a ingresar uno nuevo sale el primero que entró. Se cargan los pesos de los vehículos mientras estos sean mayor a cero. El sistema debe informar en todo momento el peso sobre el puente. Y en caso de sobrepasar los 20000 kg informar una alerta.

\item Hacer una función que recibe un vector, un número entero y devuelve un nuevo vector con los elementos del vector recibido que estan en las posiciones múltipos de número.
  
\item Una función que recibe 2 vectores A y B, la longitud se presupone que es mayor que a la del B. La función debe retornar si B está incluido en A.
  
\item Una función recibe 4 vectores: 
 \begin{enumerate}
	 \item Tiene n precios
	 \item Tiene n nombres de artículos
	 \item Tiene m números, estos números $\in [0, n)$
	 \item Tiene m cantidades	 
 \end{enumerate}
  esta función debe:
  \begin{enumerate}
	\item devolver el total
	\item Mostrar el nombre del artículo más caro de los pedidos
  \end{enumerate}

\end{enumerate}

