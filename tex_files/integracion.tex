\chapter{De Integración}

 \begin{enumerate}[resume]

  \item Una empresa necesita informatizar su facturaci\'on. Saben que cada factura tiene el nombre del cliente, el total y un n\'umero \'unico. El sistema debe:
  
  \begin{enumerate}
	  \item Permitir el ingreso de nuevas facturas. (No se puede repetir el n\'umero)
	  \item Listar las facturas dentro de un rango de n\'umeros.
	  \item Informar el total facturado.
	  \item Poder eliminar una factura.
	  \item La informaci\'on se debe conservar al cerrar y volver a abrir el sistema.
  \end{enumerate}
  
  \item Se necesita hacer una encuesta con la siguiente información:
  \begin{enumerate}  
      \item Edad del encuestado
      \item Sexo
      \item Nivel estudio: Primario, Secundario, Terciario, Universitario
      \item Ingresos promedio mensual
  \end{enumerate}
  
La encuesta es un proceso continuo que permite el reingreso al sistema varias veces para realizar la carga de los datos.
Cuando el usuario lo solicite, el sistema le deberá brindar la siguiente informacion:
	
  \begin{enumerate}
     \item Cantidad de encuestados.
     \item Porcentaje de Masculinos y femeninos.
     \item El promedio general de los ingresos
  \end{enumerate}






\end{enumerate}

